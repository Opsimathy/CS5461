\documentclass{article}
\usepackage{enumitem}
\usepackage{graphicx}
\usepackage{multirow}
\usepackage{amsmath}
\usepackage{amssymb}
\usepackage{array}
\usepackage[a4paper,margin=1in]{geometry}
\title{CS5461 Assignment 2}
\author{Li Jiaru (SID: A0332008U)}
\date{22 August 2025}
\begin{document}
\maketitle
\begin{enumerate}
\item
\begin{enumerate}[label=(\alph*)]
\item
Consider the normal-form game as given.
\begin{table}[h!]\centering\setlength{\extrarowheight}{2pt}
\begin{tabular}{|c|c|c|c|c|}
\hline
 & W & X & Y & Z \\ \hline
A & $8,6$ & $1,6$ & $2,7$ & $6,4$ \\ \hline
B & $2,4$ & $5,5$ & $4,5$ & $7,4$ \\ \hline
C & $0,2$ & $5,2$ & $3,6$ & $6,3$ \\ \hline
D & $3,5$ & $4,6$ & $3,5$ & $0,4$ \\ \hline
\end{tabular}
\end{table}

Then we notice that
\begin{itemize}
\item D is strictly dominated by $0.2\mathrm{A}+0.8\mathrm{B}$ since $8\times0.2+2\times0.8=3.2>3$ for W, $1\times0.2+5\times0.8=4.2>4$ for X, $2\times0.2+4\times0.8=3.6>3$ for Y and $6\times0.2+7\times0.8=6.8>0$ for Z,
\item W is strictly dominated by $0.5\mathrm{X}+0.5\mathrm{Y}$ since $6\times0.5+7\times0.5=6.5>5$ for A, $5\times0.5+5\times0.5=5>4$ for B, $2\times0.5+6\times0.5=3>2$ for C and $6\times0.5+5\times0.5=5.5>5$ for D,
\item Z is strictly dominated by Y since $7>4$ for A, $5>4$ for B, $6>3$ for C and $5>4$ for D.
\end{itemize}
Therefore D, W and Z are strictly dominated in the original game, but the other five actions are not.
\item
By iterated removal of dominated strategies, we only need to consider the following subgame in normal form,
\begin{table}[h!]\centering\setlength{\extrarowheight}{2pt}
\begin{tabular}{|c|c|c|}
\hline
 & X & Y \\ \hline
A & $1,6$ & $2,7$ \\ \hline
B & $5,5$ & $4,5$ \\ \hline
C & $5,2$ & $3,6$ \\ \hline
\end{tabular}
\end{table}

from which we can delete A as well since it is now strictly dominated by B or C. Therefore, we are left with
\begin{table}[h!]\centering\setlength{\extrarowheight}{2pt}
\begin{tabular}{|c|c|c|}
\hline
 & X & Y \\ \hline
B & $5,5$ & $4,5$ \\ \hline
C & $5,2$ & $3,6$ \\ \hline
\end{tabular}
\end{table}

We consider the case where the row player plays B. Then the column player will be indifferent between X and Y. When the column player plays X, the row player will be indifferent between B and C as well. However, when the row player plays C, the column player will play Y as a best response, and when the column player plays Y, the column player will play B as a best response.

Therefore we need to calculate their expected utility in each case.

For the column player, when the row player plays $(\mathrm{B},\mathrm{C})$ with probability $(p,1-p)$ where $0\le p\le 1$, we have
$$u_{\mathrm{column}}(\mathrm{X})=5p+2(1-p)=2+3p,\quad u_{\mathrm{column}}(\mathrm{Y})=5p+6(1-p)=6-p,$$
for which $2+3p=6-p\implies p=1$ is the only possibility for indifference, i.e., the row player always plays B at Nash equilibrium.

For the row player, when the column player plays $(\mathrm{X},\mathrm{Y})$ with probability $(q,1-q)$ where $0\le q\le 1$, we have
$$u_{\mathrm{row}}(\mathrm{B})=5q+4(1-q)=4+q,\quad u_{\mathrm{row}}(\mathrm{C})=5q+3(1-q)=3+2q,$$
but since the row player should always play B, we must have $4+q\ge 3+2q\implies q\le 1$ for X to be a best response.

Therefore, the Nash equilibria of the original game consist of the column player playing $(\mathrm{W},\mathrm{X},\mathrm{Y},\mathrm{Z})$ with probability $(0,q,1-q,0)$ for $0\le q\le1$ and the row player always playing B.
\end{enumerate}
\item
\begin{enumerate}[label=(\alph*)]
\item Note that when $t<2$, $(\mathrm{T},\mathrm{R})$ and $(\mathrm{B},\mathrm{L})$ are the only two pure Nash equilibria.

Similarly, when $t>2$, $(\mathrm{T},\mathrm{L})$ and $(\mathrm{B},\mathrm{R})$ are the only two pure Nash equilibria.

However, when $t=2$, all of $(\mathrm{T},\mathrm{R})$, $(\mathrm{B},\mathrm{R})$ and $(\mathrm{B},\mathrm{L})$ are the pure Nash equilibria.

Therefore, the answer is for all $t\ne 2$.
\item Note that the column player is indifferent only when the row player always plays B, since $3>2$.

Also, note that the row player is indifferent either when the row player always plays L, or when $t=2$, in which case the row player can freely choose any mixed strategy.

This is the only case in which a mixed Nash equilibrium will arise.

Therefore, the answer is also for all $t\ne 2$.
\end{enumerate}
\item
\begin{enumerate}[label=(\alph*)]
\item False.\item False.\item True.
\end{enumerate}
\end{enumerate}
\end{document}