\documentclass{article}
\usepackage{enumitem}
\usepackage{graphicx}
\usepackage{amsmath}
\usepackage{amssymb}
\usepackage[a4paper,margin=1in]{geometry}
\title{CS5461 Assignment 11}
\author{Li Jiaru (SID: A0332008U)}
\date{31 October 2025}
\begin{document}
\maketitle
\begin{enumerate}
\item
\begin{enumerate}
\item The approval counts for candidates $a,b,c,d$ are $3,4,3,2$ respectively, so AV returns the committee $\boxed{\{a,b,c\}}$ only.
\item To cover voters $4$ and $6$ we must include $c$ and $d$ respectively, and adding either $a$ or $b$ could now cover everyone. Thus CC returns either $\boxed{\{a,c,d\}}$ or $\boxed{\{b,c,d\}}$.
\item The utilities for voters $1$ to $6$ are $3,3,2,1,1,0$ respectively, so the PAV score is $2H_3+H_2+2H_1=2(1+1/2+1/3)+(1+1/2)+2=\boxed{43/6}.$
\item The initial budget for each voter is $k/n=3/6=1/2$. We pick $b$ first who has the highest approval count. Voters $1,2,3,5$ each pays $1/4$, leaving each of them $1/4$. Then $a$ is not affordable ($3\times1/4<1$), $c$ would require voters $1$ and $2$ to pay $1/4$ each and voter $4$ to pay $1/2$, and $d$ is not affordable ($1/4+1/2<1$). Thus we pick $c$. Now neither $a$ nor $d$ is affordable, so we fill the one remaining seat by approval score, and we pick $a$ since $3>2$. Thus MES returns the committee $\boxed{\{a,b,c\}}$.
\end{enumerate}
\item We prove by contradiction using a similar argument for CC satisfying JR as in lecture.

Let $S$ be a cohesive group of voters that is unrepresented by the GreedyCC committee $W$, and let $x$ be a candidate approved by all voters in $S$.

Since GreedyCC always picks a candidate that maximises the number of uncovered voters, the candidate it picks in each round must have at least $|S|$ uncovered supporters, none of which is in $S$.

Thus after $k$ rounds, it will cover at least $k|S|$ voters outside $S$, but there are only $n-|S|$ voters outside $S$ in total.

By definition of a cohesive group, we require $|S|\ge n/k$, so $k|S|\ge n>n-|S|$, giving a contradiction.
\item Let $q:=n/k$ which is an integer as $n$ is divisible by $k$. As $S$ is $t$-cohesive, by definition we have $|S|\ge tq$.

Define $S_j:=\{i\in S\mid u_i(W)<j\}$ for each $j\in\{1,2,\dots,t\}$. We claim that $|S_j|<jq$. Otherwise, we have $|S_j|\ge jq$ and each $S_j$ shares at least the same $t\ge j$ common approved candidates as $S$, so $S_j$ is a $j$-cohesive group. But as $W$ is EJR, that requires $u_i(W)\ge j$ for some $i\in S_j$, contradicting with the definition.

Recall that the \textit{Iverson bracket} $[P]$ is defined as $[P]=1$ if $P$ is true and $0$ if $P$ is false. By definition, for $x,t\in\mathbb{N}$, we have the `layer cake representation' $$\min(x,t)=\sum_{j=1}^t[x\ge j],$$ since we add $1$ for each `level' $j=1,2,\dots,t$ that $x$ reaches and stop at $t$.

By definition, $u_i(W)\in\mathbb{N}$ and $u_i(W)\ge\min(u_i(W),t)$, so we can rewrite the LHS of the required inequality as
\begin{align*}
\frac{1}{|S|}\sum_{i\in S}u_i(W)&\ge\frac{1}{|S|}\sum_{i\in S}\min(u_i(W),t)\\
&=\frac{1}{|S|}\sum_{i\in S}\sum_{j=1}^{t}[u_i(W)\ge j]\\
&=\frac{1}{|S|}\sum_{j=1}^{t}\sum_{i\in S}[u_i(W)\ge j],
\end{align*}
where we exchanged summation which is allowed since both sums are finite.

Now define the set $S'_j:=\{i\in S\mid u_i(W)\ge j\}$, which has cardinality precisely $$\lvert S'_j\rvert=\lvert\{i\in S\mid u_i(W)\ge j\}\rvert=\sum_{i\in S}[u_i(W)\ge j],$$
and by definition we also have $S'_j=S\setminus S_j$, so $\lvert S'_j\rvert=\lvert S\rvert-\lvert S_j\rvert$.

Thus we have
\begin{align*}
\frac{1}{|S|}\sum_{j=1}^{t}\sum_{i\in S}[u_i(W)\ge j]&=\frac{1}{|S|}\sum_{j=1}^{t}\lvert\{i\in S\mid u_i(W)\ge j\}\rvert\\
&=\frac{1}{|S|}\sum_{j=1}^{t}\left(\lvert S\rvert-\lvert S_j\rvert\right)\\
&=\frac{1}{|S|}\left(t\lvert S\rvert-\sum_{j=1}^{t}\lvert S_j\rvert\right),
\end{align*}

and since $\lvert S_j\rvert<jq$ and $\lvert S\rvert\ge tq\implies q/|S|\le 1/t$ as shown before, we finally get
\begin{align*}
\frac{1}{|S|}\left(t\lvert S\rvert-\sum_{j=1}^{t}\lvert S_j\rvert\right)&>t-\frac{1}{|S|}\sum_{j=1}^t jq\\
&=t-\frac{q}{|S|}\frac{t(t+1)}{2}\\
&\ge t-\frac{1}{t}\frac{t(t+1)}{2}\\
&=\frac{2t}{2}-\frac{t+1}{2}=\frac{t-1}{2},
\end{align*}

which is the required inequality.
\end{enumerate}
\bigskip
[\textbf{AI Tool Declaration}: I used ChatGPT 5 to formalise my proof idea and improve the expression for Question 3 only. I am responsible for the content and quality of the submitted work.]
\end{document}