\documentclass{article}
\usepackage{enumitem}
\usepackage{graphicx}
\usepackage{multirow}
\usepackage{amsmath}
\usepackage{amssymb}
\usepackage{array}
\usepackage[a4paper,margin=1in]{geometry}
\title{CS5461 Assignment 3}
\author{Li Jiaru (SID: A0332008U)}
\date{2 September 2025}
\begin{document}
\maketitle
\begin{enumerate}
\item
\begin{enumerate}[label=(\alph*)]
\item The VCG mechanism allocates 2 durians to Alice and 1 durian to Bob.
\item With Alice present, Bob only gets 1 durian so his utility is 3. If Alice is absent, Bob gets all 3 durians, so his utility becomes 7. Therefore Alice should be charged with $7-3=4$.

With Bob present, Alice only gets 2 durians so her utility is 6. If Bob is absent, Alice gets all 3 durians, so her utility becomes 8. Therefore Bob should be charged with $8-6=2$.
\item The VCG mechanism allocates all 3 durians to Cindy.
\item With Cindy present, both Alice and Bob get no durians so their utility is 0. If Cindy is absent, from previous parts we know that the optimal allocation guarantees a utility of $3+6=9$. Therefore Cindy should be charged with $9-0=9$.

Since both Alice and Bob get nothing, they should pay nothing either.
\item No, this is impossible. As given in the lecture, in the VCG mechanism, truthful reporting is a dominant strategy. Submitting false valuations will only make at most as much as truthful reporting does, so strict improvement is not possible.
\end{enumerate}
\item No, not necessarily. We consider a simple counterexample. Let us denote $v_1$ and $v_2$ to be the value for bidder 1 and 2, respectively, with $v_1>v_2>0$.

Under truthful valuations, bidder 1 submits $v_1$ and bidder 2 submits $v_2$, so bidder 1 will win and pay $v_2$, giving a total utility of $v_1-v_2$.

However, if they could cooperate, bidder 2 could submit a bid of $0$ and bidder 1 could then submit any positive bid. In this case, bidder 1 will still win but they pay nothing, giving a total utility of $v_1-0=v_1$.

Since $v_2>0$, $v_1>v_1-v_2$ so truthful reporting is no longer dominant.
\item No, not necessarily. We consider a simple counterexample. Suppose that there are three bidders. Your value for the item is 2, and the other two bids are 3 and 1.

Under truthful bidding, you lose and get a utility of 0.

However, you can bid anything greater than 3 and get the item, so that you only pay the third-highest bid, which is 1 in this case, and your utility becomes $2-1=1$. Therefore, the auction is no longer truthful.
\end{enumerate}
\end{document}