\documentclass{article}
\usepackage{enumitem}
\usepackage{graphicx}
\usepackage{multirow}
\usepackage{amsmath}
\usepackage{amssymb}
\usepackage{array}
\usepackage[a4paper,margin=1in]{geometry}
\title{CS5461 Assignment 6}
\author{Li Jiaru (SID: A0332008U)}
\date{20 September 2025}
\begin{document}
\maketitle
\begin{enumerate}
\item
\begin{enumerate}[label=(\alph*)]
\item We are given the game $[5; 1,2,2,4]$, so $4!=24$ permutations in total. In all permutations:
\begin{itemize}
\item Player 1 is pivotal only in the cases $(4,1,2,2)$ and $(2,2,1,4)$, each of which having 2 permutations. Therefore $\mathrm{Sh}_1=4/24=1/6$.
\item Player 4 is pivotal unless they are first or last, so $\mathrm{Sh}_4=1/2$.
\item Players 2 and 3 are symmetric, so by efficiency $\mathrm{Sh}_2=\mathrm{Sh}_3=(1-1/6-1/2)/2=1/6$.
\end{itemize}
Therefore the Shapley values are $\boxed{(1/6,1/6,1/6,1/2)}$.
\item Here each boy always adds 1 and each girl always adds 2 regardless of the coalition. Hence each player’s marginal contribution is constant. By definition, the Shapley values are $\boxed{(1,1,1,2,2,2)}$ for the three boys and the three girls.
\item There are $7!=5040$ permutations in total. In all permutations:
\begin{itemize}
\item Players 3 to 7 are symmetric and pivotal only when they are third and the first two are players 1 and 2 in any order, giving $2\times4!=48$ permutations. Therefore $\mathrm{Sh}_i=48/5040=1/105$ for $i\in\{3,4,5,6,7\}$.
\item Players 1 and 2 are symmetric, so by efficiency $\mathrm{Sh}_1=\mathrm{Sh}_2=(1-6\times1/105)/2=10/21$.
\end{itemize}
Therefore the Shapley values are $\boxed{(10/21,10/21,1/105,1/105,1/105,1/105,1/105)}$.
\end{enumerate}
\item
\begin{enumerate}[label=(\alph*)]
\item No.
\item Yes.
\item No.
\end{enumerate}
\item
\begin{enumerate}[label=(\alph*)]
\item False.
\item True.
\item False.
\end{enumerate}
\end{enumerate}
\end{document}