\documentclass{article}
\usepackage{enumitem}
\usepackage{graphicx}
\usepackage{multirow}
\usepackage{amsmath}
\usepackage{amssymb}
\usepackage{array}
\usepackage[a4paper,margin=1in]{geometry}
\title{CS5461 Assignment 1}
\author{Li Jiaru (SID: A0332008U)}
\date{22 August 2025}
\begin{document}
\maketitle
\begin{enumerate}
\item
\begin{enumerate}[label=(\alph*)]
\item
The normal-form game is shown as below.
\begin{table}[h!]\centering\setlength{\extrarowheight}{2pt}\begin{tabular}{cccc}&&\multicolumn{2}{c}{Don}\\\cline{2-4}\multicolumn{1}{c|}{}&\multicolumn{1}{c|}{}&\multicolumn{1}{c|}{Tennis}&\multicolumn{1}{c|}{Basketball}\\\cline{2-4}\multicolumn{1}{c|}{\multirow{2}{*}{Carla}}&\multicolumn{1}{c|}{Tennis}&\multicolumn{1}{c|}{$7,3$}&\multicolumn{1}{c|}{$1,1$}\\\cline{2-4}\multicolumn{1}{c|}{}&\multicolumn{1}{c|}{Basketball}&\multicolumn{1}{c|}{$1,1$}&\multicolumn{1}{c|}{$3,7$}\\\cline{2-4}\end{tabular}\end{table}
\item
$(\text{Tennis},\text{Tennis})$ and $(\text{Basketball},\text{Basketball})$ are the only two pure Nash equilibria.
\item
We only need to compute mixed equilibria here.

Therefore, we consider the case where Carla plays $(\text{Tennis},\text{Basketball})$ with probability $(p,1-p)$ and Don plays $(\text{Tennis},\text{Basketball})$ with probability $(q,1-q)$ where $0<p,q<1$.

Then each player must be indifferent between these two actions, as stated in the lecture.

For Carla, this gives
\begin{equation*}
7q+1(1-q)=1q+3(1-q),
\end{equation*}
so we have $q=1/4$.

For Don, this gives
\begin{equation*}
3p+1(1-p)=1p+7(1-p),
\end{equation*}
so we have $p=3/4$.

Therefore, Carlo playing $(\text{Tennis},\text{Basketball})$ with probability $(3/4, 1/4)$ and Don playing $(\text{Tennis},\text{Basketball})$ with probability $(1/4, 3/4)$ is a mixed Nash equilibrium.

Together with the pure Nash equilibria $(\text{Tennis},\text{Tennis})$ and $(\text{Basketball},\text{Basketball})$ as found in part (b), these give all Nash equilibria of this game.
\end{enumerate}
\item
Consider the normal-form game as given.
\begin{table}[h!]\setlength{\extrarowheight}{2pt}\centering\begin{tabular}{|c|c|c|c|}\hline & L & M & R \\ \hline T & $1,6$ & $1,0$ & $2,7$ \\ \hline C & $2,4$ & $2,2$ & $0,3$ \\ \hline B & $2,3$ & $1,4$ & $8,2$ \\ \hline\end{tabular}\end{table}

To find all Nash equilibria, we use iterated removal of dominated strategies.

We first notice that T is strictly dominated by $1/2(\mathrm{C}+\mathrm{B})$ as $1<1/2(2+2)$ for L, $1<1/2(2+1)$ for M, $2<1/2(0+8)$ for R. Therefore we can remove T.

After that, notice that R is now strictly dominated by L as $3<4$ for C and $2<3$ for B. Therefore we can remove R.

We are now left with the game in normal form as
\begin{table}[h!]\setlength{\extrarowheight}{2pt}\centering\begin{tabular}{|c|c|c|}\hline & L & M \\ \hline C & $2,4$ & $2,2$ \\ \hline B & $2,3$ & $1,4$ \\ \hline\end{tabular}\end{table}

We consider the case where the row player plays C. Then the column player will play L as a best response. In this case, the row player will be indifferent between C and B. However, when the row player plays B, the column player will play M as a best response.

Therefore we need to calculate their expected utility in each case.

For the row player, when the column player plays $(\mathrm{L},\mathrm{M})$ with probability $(q,1-q)$ where $0\le q\le 1$, we have
$$u_{\mathrm{row}}(\mathrm{C})=2q+2(1-q)=2,\quad u_{\mathrm{row}}(\mathrm{B})=2q+1(1-q)=1+q,$$
for which $2=1+q\implies q=1$ is the only possibility for indifference, i.e., the column player always plays L at Nash equilibrium.

For the column player, when the row player plays $(\mathrm{C},\mathrm{B})$ with probability $(p,1-p)$ where $0\le p\le 1$, we have
$$u_{\mathrm{column}}(\mathrm{L})=4p+3(1-p)=3+p,\quad u_{\mathrm{column}}(\mathrm{M})=2p+4(1-p)=4-2p,$$
but since the column player should always play L, we must have $3+p\ge 4-2p\implies p\ge 1/3$ for L to be a best response.

Therefore, the Nash equilibria of the original game consist of the row player playing $(\mathrm{T},\mathrm{C},\mathrm{B})$ with probability $(0,p,1-p)$ for $1/3\le p\le1$ and the column player always playing L.
\item
\begin{enumerate}[label=(\alph*)]
\item Yes.\item No.\item Yes.
\end{enumerate}
\end{enumerate}
\end{document}