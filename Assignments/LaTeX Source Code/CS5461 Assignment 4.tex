\documentclass{article}
\usepackage{enumitem}
\usepackage{graphicx}
\usepackage{multirow}
\usepackage{amsmath}
\usepackage{amssymb}
\usepackage{array}
\usepackage[a4paper,margin=1in]{geometry}
\title{CS5461 Assignment 4}
\author{Li Jiaru (SID: A0332008U)}
\date{10 September 2025}
\begin{document}
\maketitle
\begin{enumerate}
\item
\begin{enumerate}[label=(\alph*)]
\item This is truthful. For any player, if they are already at the rightmost, their cost is zero and any misreporting is going to increase that. If they are not, misreporting their location to anywhere to the left of the rightmost player is not going to change the outcome, while misreporting to become the rightmost player gives even larger cost than before, as the distance becomes farther.
\item This is not truthful. Consider a counterexample. Let the four players $1,2,3,4$ have true locations $x=0,1,3,4$ respectively. Then the facility is located at $x=2$, and player 2 bears cost $1$. Now if player 2 misreports to $x=-1$, the facility will instead be located at $x=1.5$, and player 2 bears a lower cost $0.5$.
\item This is truthful. By part (a), locating at the rightmost is truthful. Similarly, locating at the leftmost is also truthful. Hence the mechanism that mixes over these two is also going to be truthful.
\item This is not truthful. Consider a counterexample. Let the four players $1,2,3,4$ have true locations $x=0,4,6,7$ respectively. Then $a=4-0=4$ and $b=6-4=2$, and since $4>2$ the facility is located at $x=7$, so that player 1 bears a cost of 7. However, if player 1 misreports to $x=3$, $a$ becomes $4-3=1$ and the facility is instead located at $x=3$, giving a lower cost 3 for player 1.
\end{enumerate}
\item
\begin{enumerate}[label=(\alph*)]
\item Let $x$ be the flow on the upper path $s\to a\to t$, so that $1-x$ is the flow on the lower path $s\to b\to t$. Then at equilibrium, we need $c_{sa}(x)+c_{at}(x)=c_{sb}(1-x)+c_{bt}(1-x)$, so $x+1+2=2+(1-x)+1\implies x=\displaystyle\frac{1}{2}$. Thus the amount of traffic routed on the edge at equilibrium from $s$ to $a$ is $\boxed{\frac{1}{2}}$.
\item From part (a), at equilibrium the total cost is given by $\displaystyle x(c_{sa}(x)+c_{at}(x))+(1-x)(c_{sb}(1-x)+c_{bt}(1-x))=\frac{1}{2}\left(\frac{3}{2}+2\right)+\frac{1}{2}\left(2+\frac{3}{2}\right)=\boxed{\frac{7}{2}}$.
\item Note that the paths $s\to b$ and $a\to t$ are now dominated by $s\to a\to b$ and $a\to b\to t$ respectively, since for all $x<1$, we have $x+1+0<2$. Therefore in this case, all traffic will take the $s\to a\to b\to t$ path at equilibrium, and the amount of traffic routed on the edge from $s$ to $a$ is $\boxed{1}$.
\item The cost is easily computed as $1\times(1+1)+1\times0+1\times(1+1)=\boxed{4}$.
\end{enumerate}
\item The two (pure) equilibrium flows are when the top edge receives 1 unit and the bottom edge receives 2 units, and when the top edge receives 2 units and the bottom edge receives 1 unit.
\end{enumerate}
\end{document}