\documentclass{article}
\usepackage{enumitem}
\usepackage{graphicx}
\usepackage{amsmath}
\usepackage{amssymb}
\usepackage[a4paper,margin=1in]{geometry}
\title{CS5461 Assignment 9}
\author{Li Jiaru (SID: A0332008U)}
\date{17 October 2025}
\begin{document}
\maketitle
\begin{enumerate}
\item 
\begin{enumerate}
\item Agent 4's value is calculated as
\begin{align*}
v_4\left(\frac{1}{3}, \frac{2}{3}\right)&=\int_\frac{1}{3}^\frac{2}{3}f_4(x)\mathrm{d}x\\
&=\int_\frac{1}{3}^\frac{1}{2}f_4(x)\mathrm{d}x+\int_\frac{1}{2}^\frac{2}{3}f_4(x)\mathrm{d}x\\
&=0+\int_\frac{1}{2}^\frac{2}{3}(8x-4)\mathrm{d}x=\boxed{\frac{1}{9}}.
\end{align*}
\item For the Dubins–Spanier protocol, we consider, in each round, which agent shouts first (i.e., first attains a value of $1/4$).
\begin{itemize}
\item Round 1: agents 1 to 4 will shout at $x=\dfrac 1 4, \dfrac 1 8, \dfrac 1 {20}, \dfrac 3 4$ respectively. Therefore we give $\left[0,\dfrac1{20}\right]$ to agent 3 and continue.
\item Round 2: agents 1, 2, 4 will shout at $x=\dfrac 3 {10}, \dfrac 7 {40}, \dfrac 3 4$ respectively. Therefore we give $\left[\dfrac 1 {20}, \dfrac7{40}\right]$ to agent 2 and continue.
\item Round 3: agents 1 and 4 will shout at $x=\dfrac {17} {40}, \dfrac 3 4$ respectively. Therefore we give $\left[\dfrac 7{40}, \dfrac {17}{40}\right]$ to agent 1 and the rest to agent 4.
\end{itemize}
In conclusion, agents 1 to 4 get $\boxed{\left[\dfrac 7 {40}, \dfrac {17}{40}\right], \left[\dfrac 1{20}, \dfrac 7{40}\right], \left[0,\dfrac 1{20}\right], \left[\dfrac {17}{40}, 1\right]}$ respectively.
\item For the Even–Paz protocol, we consider, in each round, where each agent marks the `midpoint' for them.
\begin{itemize}
\item Round 1: agents 1 to 4 mark at $x=\dfrac 1 2,\dfrac 1 4,\dfrac 1{10},\dfrac{\sqrt{2}+2}{2}$ respectively. Therefore the left group consists of agents 2, 3 on $\left[0,\dfrac 1 4\right]$ and the right group consists of agents 1, 4 on $\left[\dfrac 1 4,1\right]$.
\item Round 2: for the left group, agents 2 and 3 mark at $x=\dfrac 1 8,\dfrac 1 {10}$ respectively. Therefore we give $\left[0,\dfrac 1{10}\right]$ to agent 3 and the rest to agent 2.
\item Round 3: for the right group, agents 1 and 4 mark at $x=\dfrac 5 8, \dfrac{\sqrt{2}+2}{2}$ respectively. Therefore we give $\left[\dfrac{1}{4},\dfrac 5{8}\right]$ to agent 1 and the rest to agent 4.
\end{itemize}
In conclusion, agents 1 to 4 get $\boxed{\left[\dfrac 1 4, \dfrac{5}{8}\right], \left[\dfrac 1{10}, \dfrac 1{4}\right], \left[0,\dfrac 1{10}\right], \left[\dfrac 5 {8}, 1\right]}$ respectively.
\end{enumerate}
\item No. Consider a counterexample where agents 1 and 2 have density functions
\[
f_1(x) = 
\begin{cases}
2,&x \in \left[0, \dfrac 1 4\right]\cup\left[\dfrac 1 2,\dfrac 3 4\right],\\
0,&\text{otherwise};
\end{cases}
\quad\text{and}\quad
f_2(x) = 
\begin{cases}
2,&x \in \left[\dfrac 1 4, \dfrac 1 2\right]\cup\left[\dfrac 3 4,1\right],\\
0,&\text{otherwise}.
\end{cases}
\]
Under the cut-and-choose protocol, agent 1 will cut at $x=\dfrac 1 2$, leaving a value of $\dfrac1 2$ for both agents. However, we could have allocated $\left[0, \dfrac 1 4\right]\cup\left[\dfrac 1 2,\dfrac 3 4\right]$ to agent 1 and $\left[\dfrac 1 4, \dfrac 1 2\right]\cup\left[\dfrac 3 4,1\right]$ to agent 2, giving each a value of 1. Therefore the protocol is not always Pareto optimal.
\item The answer is $c=\boxed{2/3}$.

We first show that the envy is always at most $2/3$. Notice that whenever an agent takes a piece, every other remaining agent values that piece by at most $1/6$, since otherwise they would have shouted first.

For any two agents, we always have one of the three cases:
\begin{itemize}
\item Both get their pieces by shouting. Then both are envy-free since each gets a piece with value at least $1/6$ and values the other's piece at most $1/6$.
\item One gets by shouting and another gets the remaining cake (or nothing), say $i$ and $j$ respectively. Then $i$ values $j$'s piece by at most $1-1/6=5/6$, so $i$ will envy $j$ by at most $5/6-1/6=2/3$.
\item Both remained till the end. Then one's envy on another is at most $1/6$ since whatever they get has value at most $1/6$.
\end{itemize}
In all cases the envy is indeed always at most $2/3$. We now show an example where the envy is at least $2/3$. Consider an example where agents 1, 2 and 3 have density functions
\[
f_1(x) = 
\begin{cases}
\dfrac 1 2,&x \in \left[0, \dfrac 1 3\right],\\
\dfrac 5 4,&x \in \left(\dfrac 1 3, 1\right];
\end{cases}
\quad
f_2(x) = 
\begin{cases}
0,&x \in \left[0, \dfrac 1 3\right],\\
\dfrac 1 2,&x \in \left(\dfrac 1 3, \dfrac 2 3\right],\\
\dfrac 5 2,&x \in \left(\dfrac 2 3, 1\right];
\end{cases}
\quad
f_3(x) = 
\begin{cases}
0,&x \in \left[0, \dfrac 2 3\right],\\
3,&x \in \left(\dfrac 2 3, 1\right].
\end{cases}
\]
Note that agents 1, 2 and 3 will shout at $x=1/3, 2/3, 1$ respectively, so each agent gets exactly $1/3$ of the original cake. Note also that agent 1 envies agent 3 by $5/4\times 1/3-1/2\times 1/3=2/3$.

In conclusion, we have shown that, in general, in the output of this protocol, any agent always envies any other agent by at most $2/3$, and this constant is indeed optimal.

(In fact, by the same argument, we could show that $c=\max(t, 1-2t)$ for any `threshold' $t$; here $t=1/6$. Since $0\le t\le 1$, $c$ is minimised at $t=1/3$ with $c_{\min}=1/3$. Therefore, the constant $1/3$ used in lecture is indeed optimal in some sense.)
\end{enumerate}
\end{document}